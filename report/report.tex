\documentclass[a4paper, 11pt]{article}
\usepackage{multirow}
\usepackage{soul}
\usepackage{graphicx}
\usepackage{amsmath}
\usepackage[export]{adjustbox}
\usepackage{float}
\usepackage[margin=1in]{geometry}
\usepackage{gensymb}
\usepackage{listings}
\usepackage{indentfirst}
\usepackage{xcolor}
% \usepackage[draft,nosingleletter]{impnattypo}

\usepackage{xcolor}

\definecolor{codegreen}{rgb}{0,0.6,0}
\definecolor{codegray}{rgb}{0.5,0.5,0.5}
\definecolor{codepurple}{rgb}{0.58,0,0.82}
\definecolor{backcolour}{rgb}{0.95,0.95,0.92}

\lstdefinestyle{mystyle}{
    backgroundcolor=\color{backcolour},   
    commentstyle=\color{codegreen},
    keywordstyle=\color{blue},
    numberstyle=\tiny\color{codegray},
    stringstyle=\color{codepurple},
    basicstyle=\ttfamily\footnotesize,
    breakatwhitespace=false,         
    breaklines=true,                 
    captionpos=b,                    
    keepspaces=true,                 
    numbers=left,                    
    numbersep=5pt,                  
    showspaces=false,                
    showstringspaces=false,
    showtabs=false,                  
    tabsize=2
}

\lstset{style=mystyle}

\begin{document}

\begin{center}
	\begin{tabular}{cc}
		\hline
		\multicolumn{2}{|c|}{\begin{tabular}[c]{@{}c@{}} \\ \LARGE \so{Informatyka w Medycynie - Laboratorium} \\ \\ \end{tabular}}         \\ \hline
		\multicolumn{2}{|c|}{\begin{tabular}[l]{@{}l@{}}\\\Large Wykrywanie naczyń dna siatkówki oka - projekt \\ \\ \end{tabular}}                      \\ \hline
		\multicolumn{1}{|l|}{\begin{tabular}[l]{@{}l@{}} \\ \hspace{2cm}Kierunek/semestr: Informatyka/6 \hspace{2cm} \\ \\ \end{tabular}} &
		\multicolumn{1}{|c|}{\begin{tabular}[l]{@{}l@{}} \\ Grupa: L16 \\ \\ \end{tabular}}                                                 \\ \hline
		\multicolumn{2}{|c|}{\begin{tabular}[c]{@{}c@{}}\\ Jakub Kwiatkowski 145356 \\Paweł Strzelczyk 145217  \\ \\ \end{tabular}}         \\ \hline
	\end{tabular}
\end{center}

\section{Opis projektu.}

Projekt został przygotowany w formie interaktywnego notatnika Jupyter Notebook.



Do wykonania symulacji wykorzystano język Python 3 oraz biblioteki

\begin{itemize}
	\item numpy
	\item matplotlib
	\item skimage
	\item OpenCV
	\item pandas
	\item ipython (ipywidgets, IPython)
	\item scikit-learn
	\item joblib
	\item tensorflow
\end{itemize}

\section{Opis wykorzystanych metod wykrywania naczyń dna siatkówki oka.}
\subsection{Przetwarzanie obrazu.}
\paragraph{Algorytm przetwarzania.}
\paragraph{Uzasadnienie.}


\subsection{Uczenie maszynowe - klasyfikator kNN.}
\paragraph{Podział obrazu na wycinki.}
\paragraph{Ekstrakcja cech.}
\paragraph{Metoda uczenia maszynowego.}
\paragraph{Ocena działania klasyfikatora na zbiorze hold-out.}
\paragraph{Uzasadnienie.}
\subsection{Przygotowanie danych.}
\paragraph{Struktura sieci.}
\paragraph{Uzasadnienie.}

\section{Wyniki.}


\section{Analiza porównawcza.}




\end{document}